\documentclass[14pt]{beamer}
\usepackage[T2A]{fontenc}
\usepackage[utf8]{inputenc}
\usepackage[english,russian]{babel}
\usepackage{amssymb,amsfonts,amsmath,mathtext}
\usepackage{cite,enumerate,float,indentfirst}
\usepackage{booktabs}
\usepackage{listings}

\graphicspath{{images/}}

\usetheme{Pittsburgh}
\usecolortheme{whale}

\setbeamercolor{footline}{fg=blue}
\setbeamertemplate{footline}{
	\leavevmode%
	\hbox{%
		\begin{beamercolorbox}[wd=.333333\paperwidth,ht=2.25ex,dp=1ex,center]{}%
			Кривчанский Н., Куликов А., ABBYY
		\end{beamercolorbox}%
		\begin{beamercolorbox}[wd=.333333\paperwidth,ht=2.25ex,dp=1ex,center]{}%
			Москва, 2017
		\end{beamercolorbox}%
		\begin{beamercolorbox}[wd=.333333\paperwidth,ht=2.25ex,dp=1ex,right]{}%
			Стр. \insertframenumber{} из \inserttotalframenumber \hspace*{2ex}
	\end{beamercolorbox}}%
	\vskip0pt%
}

\newcommand{\itemi}{\item[\checkmark]}

\title{Zadolba.li}
\subtitle{\footnotesize{Корпус текстов}}
\author{\small{%
		~Кривчанский Н., Куликов А.}\\%
	\vspace{30pt}%
	ABBYY%
	\vspace{20pt}%
}
\date{\small{Москва, 2017}}

\begin{document}
	
	\maketitle

\begin{frame}[fragile]
	
	\frametitle{Вид корпуса}
	
	\begin{itemize}
		\item Русскоязычный корпус;
		\item Широкий спектр тематик;
		\item Легко выделить несколько стилей;
		\item Имеется премодерация и теги историй.
	\end{itemize}

\end{frame}

\begin{frame}[fragile]
	
	\frametitle{Собранный корпус}
	
	\begin{itemize}
		\item 23487 истории;
		\item С 10.09.2009 по 23.10.2017;
		\item Id, название, дата публикации, теги, текст, лайки, url, ссылки на другие истории;
		\item SQLite размер 91,2 MB.
	\end{itemize}
	
\end{frame}

\begin{frame}[fragile]
	
	\frametitle{Простые статистики}
	
	\begin{itemize}
		\item 23487 истории;
		\item 4041730 слов (nltk Regexp($\backslash$w+) tokenizer);
		\item Размер истории $L \in [10; 1399]$ слов, $\bar L = 270$ слов;
		\item 464750 предложений (nltk PunktSentenceTokenizer with russian model);
		\item Размер истории $S \in [1; 173]$ предложений, $\bar S = 20$ предложений.
	\end{itemize}
	
\end{frame}

\begin{frame}[fragile]
	
	\frametitle{Распределение тегов}
	
	\begin{center}
		Самые частые теги:
		
		\begin{tabular}{c|c}
			women & 2793 \\
			transport & 2055 \\
			\colorbox{blue!30}{men} & 1566 \\
			friends & 1544 \\
			internet & 1518 \\
			healthcare & 1516 \\
			education & 1461 \\
			leisure & 1440 \\
			\colorbox{blue!30}{kids} & 1415 \\
			\colorbox{blue!30}{relatives} & 1303 \\
		\end{tabular}
	\end{center}
	
\end{frame}

\begin{frame}[fragile]
	
	\frametitle{Распределение лайков}
	
	\begin{figure}
		\includegraphics[height=110]{images/likes.png}
	\end{figure}
	
\end{frame}


\begin{frame}[fragile]
	
	\frametitle{Самые лайкаемые теги}
	
	\begin{center}
		Самые лайкаемые теги:
		
		\begin{tabular}{c|c}
			cops & 3927.01 \\
			\colorbox{blue!30}{kids} & 3668.98 \\
			religion & 3404.57 \\
			\colorbox{blue!30}{men} & 3380.20 \\
			\colorbox{blue!30}{relatives} & 3329.33 \\
			household & 3322.93 \\
			neighbors & 3223.52 \\
			state & 3217.17 \\
			golden-age & 3111.95 \\
			animals & 3088.71 \\
		\end{tabular}
	\end{center}
	
\end{frame}

\begin{frame}[fragile]
	
	\frametitle{Самые лайкаемые теги}
	
	\begin{figure}
		\includegraphics[height=150]{images/tags_by_likes.png}
	\end{figure}

	Цвет -- среднее кол-во лайков у тега.
	
	Высота -- кол-во историй у тега.
	
\end{frame}

\begin{frame}[fragile]
	
	\frametitle{Морфоразметка}
	
	\begin{itemize}
		\item Хранится в отдельной табличке;
		\item Лемма + Morpho-tag + id истории + координаты в тексте.
	\end{itemize}
	
\end{frame}

\begin{frame}
	\center{\huge
		Спасибо
	}
\end{frame}

\end{document} 